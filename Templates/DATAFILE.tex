% Имя студента
\newcommand{\student}{Бухнер Марк Евгеньевич} % полное имя студента
\newcommand{\studentshort}{Бухнер~М.~Е.} % имя студента с инициалами
\newcommand{\studentfortitle}{Бухнера Марка Евгеньевича} % имя студента в родительном падеже 
\newcommand{\studentfortask}{Бухнеру Марку Евгеньевичу} % имя студента в дательном падеже 
\newcommand{\studentforabstract}{Бухнером Марком Евгеньевичем} % имя студента в творительном падеже 

% Сведения о университете
\newcommand{\group}{20214} % номер группы
\newcommand{\department}{Кафедра систем информатики} % название кафедры
\newcommand{\departmentshort}{каф. ТК ММФ НГУ} % сокращенное название кафедры
\newcommand{\university}{Новосибирский государственный университет}
\newcommand{\universitylong}{Новосибирский национальный исследовательский государственный университет}
\newcommand{\universityshort}{НГУ}
\newcommand{\faculty}{Факультет информационных технологий}
\newcommand{\programnum}{09.03.01} % номер программы обучения
\newcommand{\programname}{Информатика и вычислительная техника}  % название программы обучения
\newcommand{\profile}{Компьютерные науки и системотехника} % профиль программы обучения
\newcommand{\administration}{Министерство науки и высшего образования РФ\\Федеральное государственное автономное образовательное\\учреждение высшего образования}

% Сведения о работе
\newcommand{\topic}{Исследование и внедрение фильтров для ускорения запросов к распределенной базе транзакционных данных}
\newcommand{\keywords}{базы данных, распределенные системы, фильтры, индексы, структуры данных, ускорение запросов, точечные запросы, интервальные запросы, SQL.}

% Распоряжение проректора по учебной работе об утверждении ВКР
\newcommand{\approvedby}{0377}                              % Номер
\newcommand{\dateapprovedby}{23~октября~2023~г.}            % Дата
\newcommand{\edateapprovedby}{<<23>>~октября~2023~г.}       % Дата

% Срок сдачи студентом готовой работы
\newcommand{\readydate}{31~мая~2024~г.}
\newcommand{\ereadydate}{<<\makebox[0.7cm]{\hrulefill}>>~\makebox[3cm]{\hrulefill}~2024~г.}


% Заведующий кафедрой
\newcommand{\apinitials}{М.~М.} 
\newcommand{\aplastname}{Лаврентьев}
\newcommand{\apname}{\aplastname~\apinitials}
\newcommand{\apdegree}{д.ф.-м.н.}
\newcommand{\aprank}{профессор}

% Руководитель ВКР
\newcommand{\sainitials}{Р.~А.}
\newcommand{\salastname}{ван Беверн}
\newcommand{\saname}{\salastname~\sainitials}
\newcommand{\sadegree}{PhD}
\newcommand{\sarank}{доцент}
\newcommand{\sarankapprovedby}{доцент}

% Соруководитель ВКР
% \newcommand{\ssainitials}{В.~Р.}
% \newcommand{\ssalastname}{Данилко}
% \newcommand{\ssaname}{\ssalastname~\ssainitials}
% \newcommand{\ssadegree}{}
% \newcommand{\ssarank}{ассистент}

% Консультант
\newcommand{\scinitials}{И.О.}
\newcommand{\sclastname}{Фамилия}
\newcommand{\scname}{\salastname~\sainitials}

% Вспомогательное
\newcommand{\prevyear}[1][]{\ifthenelse{\equal{#1}{}}{2023 г.}{20#1{23} г.}}
\newcommand{\currentyear}[1][]{\ifthenelse{\equal{#1}{}}{2024 г.}{20#1{24} г.}}
\newcommand{\datetemplate}{<<\dotuline{\hspace{9mm}}>>\dotuline{\hspace{33mm}}20\dotuline{\hspace{5mm}} г.}
\newcommand{\infotemplate}{$\underset{\text{(ФИО) / (подпись)}}{\text{\dotuline{\hspace{0.5\textwidth}/\hspace{0.45\textwidth}}}}$}
\newcommand{\signature}{\makebox[3cm]{\hrulefill}}


% Задание на ВКР
\newcommand{\taskstartdata}{ускорить выполнение запросов к распределенной базе данных Apache Hudi внедрением фильтров (структур данных), внедрить один или несколько фильтров в базу данных Apache Hudi.}
\newcommand{\taskstructure}{исследование структур данных, ускоряющих запросы на диапазоне данных (запросы с условиями); внедрение избранных структур данных в базу данных Apache Hudi; проведение замеров производительности с использованием SQL-запросов бенчмарка TPC-H; сравнение данных с исходными.}

% Текст аннотации
\newcommand{\abstractdata}{Объектом исследования данной работы являются структуры данных, позволяющие отсеивать файлы, данные в которых не удовлетворяют предикату точечных или интервальных запросов. Цель работы --- ускорение точечных и интервальных запросов в контексте систем для обработки больших данных.

Распределенные базы транзакционных данных предназначены для хранения объемов данных, которые невозможно разместить на одной физической вычислительной машине. В рамках данной работы фильтры внедряются и сравниваются в рамках системы для доступа к данным Apache Hudi. Актуальность работы состоит в том, что такие системы не имеют развитых инструментов для отсеивания нерелевантных файлов. Отчего существует множество сценариев, когда такие запросы выполняются медленно. 

Результатом данной работы является внедрение структуры данных {<<Сито-индекс>>} для пропуска нерелевантных файлов, в итоге с помощью данного фильтра удалось добиться прироста скорости выполнения точечных запросов на порядки, а интервальных запросов на порядок для тех сценариев использования, когда имеющиеся в Apache Hudi фильтры для отсеивания нерелевантных файлов --- упрощенные сводки были не способны отсеивать файлы.

Областью применения такой структуры данных являются системы для доступа к данным в платформах для обработки больших данных. }
