\subsection{Требования к внедряемой структуре данных}\label{subsec:requirements}

Далее рассмотрим основные понятия, связанные с индексами, а также сформируем ряд требований к структуре данных для отсеивания нерелевантных файлов для ее внедрения в систему для доступа к данным. Индексы представляют собой структуры данных, возникшие из необходимости обеспечить быстрый поиск информации, хранящейся в базах данных. В отсутствие индексов поиск информации по всем записям, удовлетворяющим определённым критериям, требовал бы последовательного доступа к каждой записи для проверки её соответствия условиям \cite{Tree_Data_Structures_and_Efficient_Indexing_Techniques}. Для базы данных, содержащей N элементов, это потребовало бы времени порядка $O(N)$, что для современных баз данных является неэффективным.

\begin{definition}
    \textbf{Индекс} --- это структура данных, которая позволяет ускорить процесс поиска записей с определенным свойством за счёт использования дополнительной памяти и выполнения дополнительных операций записи для поддержания своей структуры \cite{Tree_Data_Structures_and_Efficient_Indexing_Techniques}.
\end{definition}

Используя это определение, можно констатировать, что фильтры для отсеивания нерелевантных файлов являются индексами. В действительности, {<<Индекс>>} является более точным термином для описания структуры данных, позволяющей ускорить условные запросы, поэтому в рамках данной работы эти термины несут одинаковый смысл, а использование конкретного определения по большей части зависит от уже существующего названия определенной структуры. 

В контексте больших данных для сравнения методов индексации обычно рассматриваются три основные характеристики данных \cite{Big_data_The_Vs}:

\begin{enumerate}
    \item \textbf{Объём данных}: Управление и индексация больших объёмов данных представляют собой наиболее очевидную проблему в данной предметной области \cite{Big_data_challenge}. Современные объёмы данных в облачных сервисах, обрабатываемые за приемлемое время, измеряются терабайтами и, как ожидается, в ближайшем будущем достигнут петабайт \cite{Big_data_issues_challenges_tools}.
    \item \textbf{Скорость изменения данных}: Данные постоянно изменяются и обновляются. Обработка данных в реальном времени, известная как «потоковая обработка», становится альтернативой традиционной пакетной обработке \cite{Big_data_challenge}. Секторы, такие как электронная коммерция, генерируют значительные объёмы данных, например, через веб-клики, которые непрерывно поступают в поток \cite{Big_data_Issues_and_challenges}. 
    \item \textbf{Разнообразие данных}: Большие данные поступают из множества источников, включая веб-страницы, веб-журналы, социальные сети, электронные письма, документы и данные от сенсорных устройств. Различия в форматах и структуре этих данных создают проблемы, связанные с их разнообразием \cite{Addressing_big_data_variety}.
\end{enumerate}

\textbf{Объём индексируемых данных} играет ключевую роль при выборе структуры данных для реализации в озерах данных. Даже если размер индекса значительно меньше объёма данных, например, в случае с индексированием данных объёмом в терабайты, размер индекса может достигать нескольких гигабайт. Это сопоставимо с объёмом оперативной памяти современных персональных компьютеров, что делает невозможным размещение таких индексов в оперативной памяти вычислительного узла. Кроме того, индекс обычно хранится в распределённой файловой системе, доступ к которой осуществляется по сети. Следовательно, загрузка индекса в оперативную память, даже если он теоретически помещается, может занять значительное время. Это делает даже самый точный и эффективный индекс неэффективным при необходимости передавать большие объёмы данных по сети. Таким образом, важно оценивать размер персистентного хранения реализуемой структуры данных \cite{Big_data_issues_challenges_tools}.

\textbf{Скорость изменения данных} влечёт за собой требования к возможности обновления индекса, включая поддержку операций обновления, добавления и удаления записей. При этом операции записи в индекс не должны сильно замедлять запись данных в таблицу. Отдельно следует упомянуть задачу индексации уже существующих наборов данных: при больших объёмах данных выполнение этой задачи за разумное время осуществимо не для каждой структуры данных.

\textbf{Разнообразие данных} в системах для доступа к данным не является большой проблемой, так как данные хранятся в формате, определённом схемой таблицы \cite{Hudi_SQL_DDL}.

Таким образом, требования к внедряемой структуре данных в систему для доступа к данным следующие:
\begin{enumerate}
    \item Индекс должен быть обновляемым, включая тот факт, что он должен поддерживать исторические запросы и операцию удаления данных.
    \item Индекс должен отвечать на интервальные запросы.
    \item Требуется индексировать большие объемы данных, поэтому индекс должен занимать мало места (более точные оценки приведены в следующем параграфе).
    \item Индекс должен реализовывать ту же стратегию управления данными, что и пользовательская таблица.
    \item Индекс должен поддерживать возможность индексации разных типов данных (Параграф \fbox{почему??}).
\end{enumerate}
