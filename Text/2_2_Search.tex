\subsection{Поиск в индексе}

Для поиска расположений файлов \fbox{ссылка что это} в которых содержатся ключи, удовлетворяющие предикату запроса, необходимо прочитать с хранилища данных сегменты, которые отвечают за за данные ключи. В первую очередь, необходимо прочитать метаданные индекса, в которых содержится информация о существующих сегментах индекса.

Далее приведен алгоритм поиска расположений для \textbf{точечного запроса} по ключу $k$:

\begin{enumerate}
    \item Список сегментов является отсортированным, соответственно за время $O(logN)$, где $N$ - количество сегментов в индексе, возможно найти сегмент, который содержит информацию о ключе $k$, используя бинарный поиск по списку сегментов.
    \item Сегмент, содержащий информацию о ключе $k$, загружается из хранилища данных.
    \item В загруженном сегменте в массиве блоков необходимо найти блок, который содержит расположение файлов, в которых присутствует ключ $k$, сделать это возможно за время $O(1)$ вычислив номер соответствующего блока: $b_{num} = \frac{k - s_{min}}{b_{size}}$.
    \item Список расположений из полученного блока --- есть множество расположений, которые необходимо включить в рассмотрение для выполнения запроса.
\end{enumerate}

Далее обработки \textbf{интервального запроса} по ключам из интервала $[l, r]$ необходимо выполнить следующие шаги:

\begin{enumerate}
    \item За время $O(1)$ найти блок, содержащий информацию о ключе $l$, аналогично алгоритму поиска расположений для точечного запроса.
    \item Далее последовательно считывать блоки, пока для очередного блока выполняется $r \leq b_{max}$.
    \item Перейти к чтению блоков из следующего сегмента, если блоки в текущем сегменте закончились, а $r \leq b_{max}$ все еще выполняется. 
\end{enumerate}

Реализация данных алгоритмов в системе для доступа к данным Apache Hudi приведена в приложении В.
