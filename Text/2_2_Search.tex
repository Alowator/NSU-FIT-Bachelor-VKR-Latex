\subsection{Поиск в индексе}

Задача поиска нужных блоков в Сито-индексе для отфильтровывания нерелевантных данных аналогична поиску блоков данных в B-дереве. Основное различие между обработкой интервального запроса и точечным запросом заключается в следующем: для точечного запроса требуется найти и прочитать один блок, тогда как для интервального запроса необходимо определить один начальный блок, содержащий ключ из предиката. Например, для предиката a>5 необходимо найти блок, содержащий ключ 5, и последовательно прочитать все блоки, в которых находятся ключи с большими значениями. Эта операция эффективна, поскольку блоки расположены последовательно, что исключает необходимость поиска отдельных блоков в структуре индекса.

Сегменты индекса маркируются минимальным и максимальным значениями ключей, содержащихся в каждом сегменте, что позволяет определить соответствующий сегмент для заданного предиката за время O(logN) с использованием алгоритма бинарного поиска.

Номер соответствующего блока внутри сегмента может быть вычислен по формуле $(key − min_key)  block_size$, где деление выполняется как 
целочисленное, key  это искомый ключ, minkey  минимальный ключ в сегменте, а blocksize известный размер блока для текущего сегмента.
