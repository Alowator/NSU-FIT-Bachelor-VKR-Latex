\section*{Введение}
\addcontentsline{toc}{section}{Введение}

Платформы для анализа больших данных состоят из множества сервисов, среди них \fbox{ссылки?} выделяются системы для непосредственной обработки данных, системы для доступа к данным и сами хранилища данных.

Хранилища данных в таких платформах являются внешними удаленными распределенными сервисами, что превращает ввод-вывод такого хранилища в одно из основных узких мест при обработке запросов. Существует множество подходов к организации файлов для их эффективного чтения, что позволяет сократить время \fbox{оптимизировать?} обработки запроса. Тем не менее, ускорение выполнения запросов возможно и другими путями, один из них --- отсеивание файлов с данными, которые не удовлетворяют предикату запроса.

Существующие методы отсеивания файлов с данными используют упрощённые сводки (например, минимальные и максимальные значения каждого атрибута) для всех файлов данных, чтобы фильтровать те файлы, в которых не содержится записей, который удовлетворяют предикату запроса. Однако при работе с реальными данными такие подходы не всегда оказываются эффективными ввиду того, что каждый файл с данными может содержать большой диапазон значений, например --- отметки времени за январь по декабрь определенного года. Тогда для запросов, нацеленных на извлечение данных за определенные периоды, например, с июня по август того же года, такой подход приведет к необходимости чтения файла, который не содержит нужной информации.

Целью данной работы является разработка оптимальных методов индексации данных для ускорения обработки запросов с условиями в контексте платформ больших данных на основе анализа существующих подходов и их применимости на практике.

Задача исследования состоит в разработке структуры данных, которая позволит ускорить обработку интервальных запросов в платформах для обработки больших данных, а также во внедрении этой структуры данных в одну из платформ с последующим проведением экспериментов на реальных наборах данных. \cite{tex, latex}
