\section*{Введение}
\addcontentsline{toc}{section}{Введение}

Конечный результат написания дипломной работы в области информационных технологий – это важный и значимый этап в образовательном процессе студента. В рамках данной работы будут рассмотрены теоретические и практические аспекты задачи совершенных раскрасок графов, а также проведен анализ существующих методов и алгоритмов решения данной проблемы.

Целью данной работы является разработка эффективного алгоритма для нахождения совершенных раскрасок графов в заданный численный диапазон цветов. В ходе работы будут использованы методы и технологии программирования на языке Python, а также пакеты компьютерной графики TikZ и PGF для создания графических представлений решений задачи.

Для достижения поставленной цели в работе будут использованы результаты предыдущих исследований в области теории графов и комбинаторики, а также современные методы анализа данных и машинного обучения. Большую роль в исследовании будет играть Message Passing Interface (MPI) – стандарт, обеспечивающий передачу сообщений между процессами в распределенной вычислительной системе.

Данная работа имеет практическое значение для различных областей, таких как математика, информационные технологии и прикладные науки. В результате выполнения данной работы будет разработан алгоритм, который может быть использован для решения широкого круга задач, связанных с проблемой совершенных раскрасок графов.

В качестве основных источников для написания данной работы были использованы книги \cite{tex, latex} и статьи \cite{augustinovich_11, horoshilova_9, puzynina_5}. Также в работе будет использована онлайн-документация MPI \cite{mpi} и пакетов компьютерной графики TikZ и PGF \cite{tikz}.