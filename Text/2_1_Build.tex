\subsection{Алгоритм построения Сито-индекса}\label{subsec:build}

Структура данных Сито-индекс, как и Sieve, состоит из сегментов, каждый из которых представляет собой отдельный файл в хранилище данных. Каждый такой файл содержит в себе множество блоков. Также отдельно вводится понятие \textbf{метаданные} Сито-индекса, это отдельный файл, в котором содержится список сегментов.

Каждый блок содержит в себе множество расположений $b.L$, в которых располагаются ключи из интервала $[b_{min}, b_{max}]$. Добавим для каждого существующего расположения $loc$ счетчик $b.L(loc)$, которым будем вести подсчет количества ключей, которые содержатся в данном расположении $loc$. Такое решение необходимо для реализации операций обновления индекса.

Далее разработаем операции над данной структурой данных, принимая ввиду ограничения платформ для обработки больших данных (см. раздел \ref{subsec:requirements}).

Алгоритм построения состоит из двух этапов: сортировка ключей и построение сегментов индекса.

Первый этап является вычислительно самым трудоемким, так как одно из ограничений в платформах для обработки больших данных --- ограниченный размер оперативной памяти вычислительных узлов (см. раздел \ref{subsec:requirements}). Для того чтобы построить функцию $R$, необходимо отсортировать весь набор ключей для того чтобы отслеживать изменение значения функции $R$. Так как размер всего множества ключей $K$ может не уместиться в оперативную память одного вычислительного узла и даже всех вычислительных узлов вместе взятых. Однако данную операцию возможно реализовать распределенно, с использованием системы для обработки данных (см. раздел \ref{subsec:definition}), то есть производить данные вычисления, используя вычислительные ресурсы всего кластера, так как система для обработки данных имеют возможность работы выполнения таких запросов как сортировка и агрегация данных, которые не умещаются в оперативную память, при этом может задействоваться локальная файловая система узлов кластера, при недостатке оперативной памяти вычислительных узлов~\cite{Impact_of_memory_size_on_bigdata}. В данной работе системой для обработки данных является Apache Spark.

Структура данных {<<Sieve>>} имеет реализованный алгоритм построения индекса, однако его реализация не рассматривает ограничение размера оперативной памяти для построения сегментов. Реализуем модифицированную версию данного алгоритма.

Сперва необходимо выполнить первый этап и получить набор данных $df$, содержащий отсортированные и агрегированные данные. Второй этап заключается в формировании сегментов и сохранении их в хранилище данных. Этот процесс происходит итеративно, получая из вычислительного кластера каждый следующий ключ вместе с множеством его расположений один за одним и отслеживая значение ошибки для текущего сегмента. Данная операция производится на одном вычислительном узле, так как для формирования сегментов индекса необходимо пройти все множество ключей по возрастанию один раз. Формирование очередного сегмента заканчивается при достижении им определенного размера (так как объем оперативной памяти вычислительного узла ограничен) или при достижении порога ошибки, согласно алгоритму для выделения сегментов. После формирования очередного сегмента, необходимо вычислить размер блоков (\ref{eq:partition_size}) внутри него и сформировать эти блоки, используя алгоритм формирования блоков (представлен далее в этом разделе), после чего сегмент сохраняется в хранилище данных и может быть выгружен из оперативной памяти. Таким образом возможно преодолеть ограничение оперативной памяти вычислительного узла и невозможность выполнить сегментацию в системе обработки данных.

\textbf{Алгоритм построения Сито-индекса:}

На входе: $indexColumn$ --- индексируемый атрибут, $locations$ --- атрибут расположения записи.

На выходе: метаданные Сито-индекса, множество построенных сегментов $S$.
\begin{enumerate}
\item $df = select(indexColumn, locations).sort(indexColumn)\\.groupBy(indexColumn).add(collect(locations))$
\item Создается пустой список $S$ для хранения метаданных сегментов индекса.
\item Итерация по набору данных $df$ из отсортированных ключей, для очередного ключа $k$ и его множества расположений $loc$:
    \begin{enumerate}
    \item Если текущий сегмент $s$ не содержит ключей, то в него добавляется $k$ и множество расположений $loc$.
    \item Если значение $sl$ достигло определенной ошибки (шаг 3 алгоритма для выделения сегментов), производится формирование блоков для этого сегмента и он сохраняется в файловое хранилище, а его метаданные в $S$, после чего он может быть выгружен из памяти.
    \item Иначе $k$ вместе с его расположениями $loc$ сохраняется в сегменте $s$ для дальнейшего формирования блоков.
    \end{enumerate}
\item Если последний строящийся $s$ содержит ключи, производится формирование блоков для этого сегмента и он сохраняется в файловое хранилище, а его метаданные в $S$.
\item В файловое хранилище сохраняются метаданные индекса, в виде списка метаданных доступных сегментов $S$, каждый элемент такого списка $S_i$ содержит лишь значения $S_{(i)min}$ и $S_{(i)max}$. Сегменты сохраняются упорядоченно $S_{(i)max} < S_{(i+1)min}$.
\end{enumerate}

\textbf{Алгоритм формирования блоков:}

На входе: выделенный сегмент $s$.

На выходе: построенный сегмент $s$ с массивом блоков $s.B$.
\begin{enumerate}
    \item Вычислить размер $b_{size}$ по формуле (\ref{eq:partition_size}).
    \item Вычислить кол-во блоков для данного сегмента $b_{count} = s_{size} / b_{size}$.
    \item Сформировать массив блоков $B$ размеров $b_{count}$.
    \item Для каждого ключа $k$ и его множества расположений $loc$: 
        \begin{enumerate}
        \item Вычислить $i = (k - s_{min}) / b_{size}$.
        \item Добавить в блок $B_{i}$ все расположения $loc$, обновить счетчики расположений $B_{i}.L$ для каждого расположения из $loc$.
        \end{enumerate}
    \item Сохранить массив блоков $B$, остальная информация может быть выгружена из памяти.
    \item Установить значение предела дополнительных ложноположительных ответов для сегмента $s.E = e \cdot b_{count}$
\end{enumerate}

Инициировать создание Сито-индекса возможно с помощью SQL-запроса, этот процесс подробно описан в руководстве пользователя (Приложение А) и описании программы (Приложение Б).
