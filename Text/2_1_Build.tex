\subsection{Построение индекса}

Задача построения индекса формулируется следующим образом: для заданного индексируемого атрибута требуется создать и реализовать персистентное хранение индекса в таблице. Важно отметить, что в общем случае индексируемый атрибут не умещается в оперативной памяти вычислительного узла.

При построении индекса работаем с отношением, содержащим два атрибута: индексируемый атрибут (ключ) и расположение файла данных, содержащего данный ключ. Уникальность ключей в общем случае не гарантируется, что делает необходимым сортировку данного набора для построения кумулятивной распределительной функции.

Процесс построения индекса различается для новых и существующих таблиц. Для новой таблицы построение индекса начинается с декларации его использования, после чего происходит обновление индекса согласно пункту 2.4.

Для существующих таблиц необходимо, чтобы набор ключей был отсортирован, причем уникальность каждого ключа не гарантируется. Простейшее решение заключается в применении любого метода сортировки, при этом стабильность сортировки не является обязательной. Сложность задачи построения индекса для существующей таблицы возрастает из-за необходимости сортировать и агрегировать большой объем данных для определения расположения каждого ключа. Проблема вытекает из необходимости создания упорядоченного массива ключей, на основе которого будет сформирована кумулятивная распределительная функция.

Построение индекса для существующей таблицы представляет собой более общий случай. Основная задача заключается в формировании упорядоченного массива ключей для создания кумулятивной распределительной функции.

Для решения этой задачи применим возможности распределенного движка Spark, используемого в Hudi. С его помощью осуществляется сортировка ключей и агрегация, что позволяет получить набор расположений для каждого уникального ключа. Этот метод является оптимальным, так как задача сортировки большого объема данных, который не умещается в оперативной или даже постоянной памяти компьютера, имеет известные решения, но остается нетривиальной.

После сортировки ключей достаточно в однопоточном режиме пройтись по каждому из них, поддерживая текущее кумулятивное значение функции для определения момента изменения тренда данных и необходимости формирования нового сегмента.

Существует вероятность, что весь набор данных будет иметь общий тренд, и с точки зрения подхода Sieve целесообразно создать один большой сегмент. Однако это невозможно из-за ограничений оперативной памяти вычислительного узла, что требует прекращения генерации сегмента при достижении его определенного размера. Экспериментально было установлено, что размер сегмента не должен превышать 3 миллиона записей при выделенных 2 ГБ оперативной памяти на вычислительный узел. Далее следует определить оптимальный размер блока, при котором вероятность ошибки окажется ниже установленного порога. Каждый сегмент делится на блоки, в которых сохраняется информация о местоположениях, ассоциированных с данным интервалом ключей. После завершения этих операций сегмент сериализуется на файловую систему.

Таким образом, благодаря возможностям распределенного движка и особенностям однопроходного алгоритма сегментации данных, можно эффективно построить Сито-индекс для объемного набора данных, который не умещается в память компьютера.
