\section{Сито-индекс и его внедрение в систему доступа к данным}

Существуют также эвристические подходы, которые используют закономерности в данных. Например, подход Sieve основан на наблюдении, что близкие по значению ключи часто находятся в одном и том же наборе данных, или распределены среди различных файлов данных [1]. Sieve использует кумулятивную распределительную функцию для определения размера следующего индексного блока: если значения функции для ключей k и k−1 указывают на одинаковые множества расположений, значение функции равно 0; в противном случае — 1. Sieve группирует множество ключей, относящихся к одному и тому же набору блоков, в отдельный сегмент. Эта идея схожа с принципом легковесных агрегаций, однако предлагает большую гибкость в построении индексов.

На данном этапе возможно синтезировать структуру данных, объединяющую преимущества различных подходов, с учетом существующих ограничений предметной области. Это позволит описать структуру, эффективно отсеивающую ненужные файлы данных и ускоряющую выполнение запросов.

Целесообразно использовать подход Sieve для выявления трендов в распределении ключей между файлами данных. К каждому полученному сегменту применяется классический метод блочных индексов. Каждый сегмент делится на блоки равной длины, содержащие информацию о расположении каждого ключа в интервале [l; r].

Для быстрого нахождения необходимого множества блоков отмечается, что при выполнении интервальных запросов в такой структуре требуется последовательное считывание расположенных подряд блоков. Для решения этой задачи в классических СУБД обычно используются B-деревья и их модификации. Оптимальная скорость сканирования в B-деревьях достигается за счет использования структуры в виде плоского дерева, где каждый узел представляет собой сегмент. Учитывая, что размер каждого блока внутри сегмента фиксирован, точечный запрос в такой структуре можно выполнить за время O(logN), а затем найти нужный блок можно за константное время. Для интервальных запросов требуется последовательное чтение следующих блоков. Такой подход к индексации данных будем называть "Сито", а реализуемую структуру данных — "Сито-индексом".
