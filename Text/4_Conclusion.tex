\section*{Заключение}
\addcontentsline{toc}{section}{Заключение}

В рамках выпускной квалификационной работы были рассмотрены и применены структуры данных, способствующие ускорению точечных и интервальных запросов в условиях распределенных баз данных, типичных для аналитических систем. Результатом работы стал синтез структуры данных Сито-индекс. Данная структура данных позволила значительно повысить эффективность обработки запросов с условиями за счет быстрого отсеивания блоков данных, которые не соответствуют заданным в запросе ограничениям. Эксперименты показали, что внедрение Sieve индекса привело к приросту производительности на аналитических запросах из бенчмарка TPC-H, подтверждая тем самым  эффективность предложенных решений.

Выпускная квалификационная работа выполнена мной самостоятельно с соблюдением правил профессиональной этики. Все использованные в работе материалы и заимствованные принципиальные положения (концепции) из опубликованной научной литературы и других источников имеют ссылки на них. Я несу ответственность за приведенные данные и сделанные выводы.

Я ознакомлен с программой государственной итоговой аттестации, согласно которой обнаружение плагиата, фальсификации данных и ложного цитирования является основанием для не допуска к защите выпускной квалификационной работы и выставления оценки «неудовлетворительно».

\vspace{3em}
\noindent
Бухнер Марк Евгеньевич \hspace*{\fill} \makebox[5cm]{\hrulefill}\\
\vspace{-3em}
\begin{flushright}
			\footnotesize (подпись)~~~~~~~~~~~~~~~
\end{flushright}

\noindent
% 31 мая 2020 г.
<<\makebox[0.7cm]{\hrulefill}>>~\makebox[3cm]{\hrulefill}~2024~г. 
