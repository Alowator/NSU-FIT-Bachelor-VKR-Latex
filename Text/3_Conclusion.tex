\section*{Заключение}
\addcontentsline{toc}{section}{Заключение}

В рамках данной выпускной квалификационной работы были рассмотрены имеющиеся структуры данных в системах для доступа к данным, способствующие ускорению точечных и интервальных запросов. Результатом работы стало внедрение структуры данных Сито-индекс, основанной на алгоритме сегментирования данных Sieve. Данная структура данных позволила значительно повысить эффективность обработки запросов с условиями за счет отсеивания блоко файлов, которые не содержат данных, которые удовлетворяют предикату запроса.

Автор данной работы не утверждает, что данное решение является оптимальным для поставленной задачи. Например, существует множество структур данных, основанных на фильтре Блума, которые потенциально могли бы решить данную задачу. Однако эксперименты показали, что внедрение структуры данных Сито-индекс привело к приросту производительности точечных и интервальных запросов.

Выпускная квалификационная работа выполнена мной самостоятельно с соблюдением правил профессиональной этики. Все использованные в работе материалы и заимствованные принципиальные положения (концепции) из опубликованной научной литературы и других источников имеют ссылки на них. Я несу ответственность за приведенные данные и сделанные выводы.

Я ознакомлен с программой государственной итоговой аттестации, согласно которой обнаружение плагиата, фальсификации данных и ложного цитирования является основанием для не допуска к защите выпускной квалификационной работы и выставления оценки «неудовлетворительно».

\vspace{3em}
\noindent
Бухнер Марк Евгеньевич \hspace*{\fill} \makebox[5cm]{\hrulefill}\\
\vspace{-3em}
\begin{flushright}
			\footnotesize (подпись)~~~~~~~~~~~~~~~
\end{flushright}

\noindent
% 31 мая 2020 г.
<<\makebox[0.7cm]{\hrulefill}>>~\makebox[3cm]{\hrulefill}~2024~г. 
